\vspace{10cm}
%\begin{center}
%\section{Flow Sensor DAQ Source Code}
%\end{center}

%What follows is the source code used to build the flow sensor data acquisition front panel. This code is to be run in a Visual %Basic Compiler, specifically VB 5.0 or 6.0 using a Win98 operating system. This code requires that RSTimer be installed in the %Visual Basic Environment. Additionally, parallel port addresses and ports should be aligned and set in the code before use.

%\pagebreak
\begin{singlespace}
\begin{small}
\begin{verbatim}
'Flow Sensor Interface V2
'Written by James Kelly
'Surface and Terrestrial Processes Lab
'April 2008
'
'This is an interface driver that reads data from LPT1
'on Data Pin 1 from a rotary flow sensor.

'RSTimer1 is for time stamp generation at 0.1 sec resolution
'RSTimer2 is for frequency calculation at 1 sec resolution
'RSTimer3 is for edge counting from the flow sensor signal at 1ms resolution
'RSTimerFlow is for flow specific time management only during flow events

Option Explicit

'Subs to establish Parallel Port commands
Private Declare Sub vbOut Lib "WIN95IO.DLL" 
(ByVal nPort As Integer, ByVal nData As Integer)
Private Declare Sub vbOutw Lib "WIN95IO.DLL" 
(ByVal nPort As Integer, ByVal nData As Integer)
Private Declare Function vbInp Lib "WIN95IO.DLL" 
(ByVal nPort As Integer) As Integer
Private Declare Function vbInpw Lib "WIN95IO.DLL" 
(ByVal nPort As Integer) As Integer

'output data arrays
Private flowData() As Double
Private timeData() As Double
Private timeStamp() As String
Private cumFlowData() As Double

'timekeeping vars
Private msec As Long, sec As Long, min As Long, hr As Long
Private fmsec As Long, fsec As Long, fmin As Long
Private tStmp As Double
Private fTime As Double, calibrationConst As Double

'operating flags
Private outToggle As Boolean
Private activeFlag As Boolean, launchTim As Date, flowON As Boolean
Private saveFlag As Boolean

'sampling vars
Private i As Long
Private freqCount As Double
Private edgeDetect As Integer, deltaEdge As Integer
Private cumF As Double, currF As Double

'File and calibration vars
Private volume As Double
Private portNumber As Integer
Private Units As Boolean 'Imperical = False, Metric = True



'Front Panel Initialization
Private Sub Form_Load()

activeFlag = False

'Clock intialize
msec = 0
sec = 0
min = 0
hr = 0

'flow clock initialize
fmsec = 0
fsec = 0
fmin = 0

tStmp = 0
et.Caption = "0:0:0.0"
sr.Text = "1"
cf.Caption = "0"
lt.Caption = ""
ft.Caption = ""
cumFlow.Caption = "0"
flowON = False
currF = 0
cumF = 0
cConstant.Text = "0.005766"
cmdStop.Enabled = False
cmdReset.Enabled = False
saveFlag = True
portNumber = 957

'initialize frequency measurement vars
edgeDetect = 0
deltaEdge = 0
freqCount = 0

'Intialize intervals of timers
RSTimer1.Interval = 100     '0.1 sec
RSTimer2.Interval = 1000    '1 sec
RSTimer3.Interval = 1       '1ms
RSTimerFlow.Interval = 100
RSTimerFlow.Enabled = False

End Sub
'Launch for DAQ
Private Sub cmdLaunch_Click()
        
        i = 0
        Call Reset
        
        RSTimer1.Enabled = True
        RSTimer2.Enabled = True
        lt.Caption = Time & "  " & Date
        activeFlag = True
        saveFlag = False
        flowON = False
        cmdCalibrate.Enabled = False
        cmdLaunch.Enabled = False
        cmdQuit.Enabled = False
        cmdStop.Enabled = True
        cmdReset.Enabled = True
        
        Do
        DoEvents
        If freqCount <> 0 Then
            
            RSTimerFlow.Enabled = True
            If flowON <> True Then fs.Caption = Time
            flowON = True
            
        End If
        If freqCount = 0 Then RSTimerFlow.Enabled = False
        Loop Until activeFlag = False
        
        cmdCalibrate.Enabled = True
        cmdLaunch.Enabled = True
        cmdQuit.Enabled = True
        cmdStop.Enabled = False
        cmdReset.Enabled = True

End Sub

'Time stamp generation
Private Sub RSTimer1_Timer()

'Nested If counts out for a time display
msec = msec + 1
If msec > 9 Then
msec = 0
sec = sec + 1
If sec > 59 Then
sec = 0
min = min + 1
If min > 59 Then
min = 0
hr = hr + 1
End If
End If
End If

'Double Int time stamp is computed for data handling
tStmp = sec + 0.001 * msec

'Assemble counters for time visualization
et.Caption = CStr(hr & ":" & min & ":" & sec & "." & msec)

'---Insert to test port 1 output response - must test ok at 1kHz---
'If outToggle = True Then
'vbOut 888, 0
'outToggle = False
'Else
'vbOut 888, 1
'outToggle = True
'End If

End Sub
'Timer for Freq calcuation at 1sec Resolution
Private Sub RSTimer2_Timer()

'Current Sample Flux Monitoring
ts.Caption = CStr(freqCount / CDbl(sr.Text))
'Reset sampling counter
freqCount = 0

'Compute Current Flow Rate
currF = 0.0164 * CDbl(ts.Caption) ^ 0.7504
'Display Current Flow
cf.Caption = CStr(currF)
If flowON = True Then

            'Compute Cumulative Flow
            cumF = currF + cumF
            'Display Cumulative Flow
            cumFlow.Caption = CStr(cumF)

            ReDim Preserve flowData(i)
            ReDim Preserve timeData(i)
            ReDim Preserve timeStamp(i)
            ReDim Preserve cumFlowData(i)
            
            RSTimerFlow.Enabled = True
            If flowON <> True Then fs.Caption = Time
            flowON = True
            
            'assembling data arrays for export
            flowData(i) = CDbl(cf.Caption)
            timeData(i) = i
            timeStamp(i) = Time
            cumFlowData(i) = CDbl(cumFlow.Caption)
            
            'visualization of data
            dataOut.Text = dataOut.Text & CStr(timeData(i))
		 & vbTab & Format$(flowData(i), "0.000") & vbCrLf
            i = i + 1

End If

End Sub
'Timer for edge counting of sensor signal at 1ms res
Private Sub RSTimer3_Timer()

edgeDetect = vbInp(portNumber)

If edgeDetect <> deltaEdge Then
    freqCount = freqCount + 1
    deltaEdge = edgeDetect
End If

End Sub

Private Sub RSTimerFlow_Timer()
'This is essentailly a duplication of the main elapsed time clock,
'but used for flow timing only.

fmsec = fmsec + 1
If fmsec > 9 Then
fmsec = 0
fsec = fsec + 1
If fsec > 59 Then
fsec = 0
fmin = fmin + 1
End If
End If

fTime = fmin + 0.016949152542 * fsec + 0.001 * fmsec

'Assemble counters for flow time visualization
If activeFlag = True Then fft.Caption = 
	CStr(fmin & ":" & fsec & "." & fmsec)
End Sub
'This resets display data and clears any data remnants in the output array.
'This is invoked manually or at launch time.
Private Sub cmdReset_Click()
Dim perg As Integer

If saveFlag = False Then
perg = MsgBox("Data has not been saved", vbOKCancel, "Data Purge?!")
    
    If perg = 2 Then Exit Sub

End If

lt.Caption = ""
et.Caption = ""
ft.Caption = ""
cf.Caption = ""
cumFlow.Caption = ""
ts.Caption = ""
fs.Caption = ""
fft.Caption = ""
fv.Caption = ""
dataOut.Text = ""
fmsec = 0
fsec = 0
fmin = 0

ReDim flowData(0)
ReDim timeData(0)
ReDim timeStamp(0)
ReDim cumFlowData(0)

saveFlag = True

cmdReset.Enabled = False

End Sub

Private Sub cmdSave_Click()

On Error GoTo DialogError

With CommonDialog1
    .CancelError = True
    .Filter = "Exported Data (*.dat)|*.dat"
    .FilterIndex = 1
    .DialogTitle = "Select a location and name for your data file"
    .ShowSave
End With

Dim c As Long
c = 0

Open CommonDialog1.filename For Output As #1

'Write data to filestream....
'Header info
Print #1, "ASU Surface and Terrestrial Processes Lab"
Print #1, "Stormwater discharge simulator flow sensor output"
Print #1, Date & timeStamp(0)
Print #1, comment.Text & vbCrLf
Print #1, "ElapTime (s)" & vbTab & "Current Flow" & 
		vbTab & "Cumulative Flow"

'dump flow data from arrays to filestream
Do
Print #1, timeStamp(c) & vbTab & timeData(c) & vbTab 
	& Format$(flowData(c), "0.00") & vbTab & Format$
		(cumFlowData(c), "0.00") & vbTab
c = c + 1
Loop While c < i

If c = 0 Then MsgBox ("No Data was Exported")

saveFlag = True
Close

DialogError:
If saveFlag = False Then MsgBox ("No Data was Exported")
End Sub

'Sub for creating a calibration curve and related linear constant
'Output calibration data is sent to a default flowCalibrate_x.dat file
'careful - this one's a mess!
Private Sub cmdCalibrate_Click()

'Calibration arrays
Dim flowDat() As Double
Dim timeDat() As Long
'For each variant variable
Dim flow As Variant
Dim timeD As Long
Dim sum As Double
Dim c As Integer, cc As Integer

Dim strVolume As String
Dim dataSave As Integer
    dataSave = 1
Dim sampleCnt As Long
    
cmdCalibrate.Enabled = False
cmdLaunch.Enabled = False
cmdQuit.Enabled = False
cmdStop.Enabled = True
cmdReset.Enabled = False

cConstant.Text = "1"
activeFlag = True
strVolume = InputBox("Enter Calibration Volume in Litres",
		 "Calibration Volume", "50")



If IsNumeric(CDbl(strVolume)) Then
    volume = CDbl(strVolume)
Else
    MsgBox ("Entered Value is not a number")
Exit Sub
End If
        
'management of existing data
If saveFlag = False Then dataSave = MsgBox("You may have unsaved data. 
	Cablibration will clear this. Continue?", vbOKCancel)
If dataSave = 1 Then
    saveFlag = True
    i = 0
Else
      Exit Sub
End If
        
RSTimerFlow.Enabled = True
        
        i = 0
        Call Reset
        
        RSTimer1.Enabled = True
        lt.Caption = Time & "  " & Date
        activeFlag = True
        saveFlag = False
        flowON = False
     
        Do
        DoEvents
        If freqCount <> 0 Then
            
            RSTimerFlow.Enabled = True
            If flowON <> True Then fs.Caption = Time
            flowON = True
            
        End If
        If freqCount = 0 Then RSTimerFlow.Enabled = False
        Loop Until activeFlag = False
        
    RSTimer1.Enabled = False
    RSTimer2.Enabled = False
    RSTimer3.Enabled = False
    RSTimerFlow.Enabled = False
    cmdEnable.Caption = "Timers Off"
    cmdEnable.Enabled = False

c = 0
cc = 0
sum = 0
dataOut.Text = ""

For Each flow In flowData
    If flow <> 0 Then
        ReDim Preserve flowDat(c)
        ReDim Preserve timeDat(c)
        flowDat(c) = flow
        timeDat(c) = timeData(cc)
        dataOut.Text = dataOut.Text & vbCrLf & timeDat(c) & 
		vbTab & Format$(flowDat(c), "0.00")
        
        sum = sum + flowDat(c)
                
        c = c + 1
    End If
    
    timeD = c
    
cc = cc + 1
Next
dataOut.Text = dataOut.Text & vbCrLf & "Time = " & CStr(timeD) & 
		vbCrLf & "Sample# = " & CStr(sum)

RSTimerFlow.Enabled = False

'calibrationConst = timeD / volume
calibrationConst = volume / sum
cConstant.Text = CStr(calibrationConst)
        
cmdCalibrate.Enabled = True
cmdLaunch.Enabled = True
cmdQuit.Enabled = True
cmdStop.Enabled = False
cmdReset.Enabled = True
        
End Sub
Private Sub cmdEnable_Click()

If (cmdEnable.Caption = "Disable") Then
    RSTimer1.Enabled = False
    RSTimer2.Enabled = False
    RSTimer3.Enabled = False
    RSTimerFlow.Enabled = False
    cmdEnable.Caption = "Timers Off"
    cmdEnable.Enabled = False
End If

End Sub

Private Sub cmdStop_Click()
RSTimer1.Enabled = False
RSTimer2.Enabled = False
activeFlag = False
flowON = False

        cmdCalibrate.Enabled = True
        cmdLaunch.Enabled = True
        cmdQuit.Enabled = True
        cmdStop.Enabled = False
        cmdReset.Enabled = True

End Sub
Private Sub cmdQuit_Click()
If saveFlag = False Then
MsgBox ("Data has not been saved. Please reset or save.")
Exit Sub
End If
End
End Sub
\end{verbatim}
\end{small}
\end{singlespace}